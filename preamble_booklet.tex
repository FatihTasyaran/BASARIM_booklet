%======================================================
% This file is part of
% "AMCOS_booklet"
% Version 1.1 (04/07/2019)
% A LaTeX template for conference books of abstracts
%
% This template is available at:
% https://github.com/maximelucas/AMCOS_booklet
%
% License: GNU General Public License v3.0
%
% Authors:
% Maxime Lucas (ml.maximelucas@gmail.com)
% Pau Clusella
%=======================================================

%
% COLORS
%

\definecolor{myorange}{RGB}{255,117,40}
\definecolor{mygray}{RGB}{164, 168, 172}
\definecolor{mywhite}{RGB}{235, 238, 231}
\definecolor{myblue}{RGB}{52, 115, 116}
%BASARIM COLORS                              
\definecolor{basarimblack}{RGB}{4,7,7}
\definecolor{basarimgold}{RGB}{183, 137, 44}
\definecolor{basarimblue}{RGB}{17,54,93}


\newcommand{\primarycolor}{basarimgold}
\newcommand{\secondarycolor}{basarimblack}
\newcommand{\ternarycolor}{basarimblue}

%
% BOOKLET VERSIONS
%

% If compilation is done with 'compile.sh', both versions (online and printed) are automatically compiled
% If compilation is done from editor, choose which version to compile below
\makeatletter
\@ifundefined{ifOnline}{% % check if already defined from the command line, if not define \ifOnline
	\expandafter\newif\csname ifOnline\endcsname
	\Onlinefalse %set to \Onlinefalse/\Onlinetrue for printed/online version
}{}
\makeatother

% define \type to input the right version of the abstracts
\ifOnline
\newcommand{\type}{o}
\else
\newcommand{\type}{p}
\fi % end if

%
% ABSTRACT ENVIRONMENTS
%

%----------------------------------------
% basarim abstract environment
%----------------------------------------
\newenvironment{abstract_basarim}[7] %{date}{track}{room}{title}{author}{affiliation}{type}
{%\filbreak %avoid page break

        \clearpage\pagestyle{fancy}
        \fancyhf{}
        \lhead{#1}
        \chead{#2}
        \rhead{#3}
        \cfoot{Sabancı Üniversitesi, Altunizade Dijital Kampüs, Istanbul}
        \rfoot{\thepage}
        
	{\Large \bfseries #4}
	
	{\bfseries \itshape #5} \newline\newline {\noindent #6}
       	\textcolor{mygray}{#7}
        %\addcontentsline{toc}{subsection}
        
}
{\clearpage\pagestyle{plain}}
%----------------------------------------
% online abstract environment
%----------------------------------------
\newenvironment{abstract_online}[4] %{title}{author}{affiliation}{type}
{\filbreak %avoid page break
	
	{\large \bfseries #1}
	
	{\bfseries \itshape #2} \hfill {#3}
	
	\textcolor{mygray}{#4}
	
}
{}

%----------------------------------------
% talk abstract environment (printed)
%----------------------------------------
\newenvironment{abstract}[4] %{title}{author}{affiliation}
{\filbreak %avoid page break
	
	{\large \bfseries #1}
	
	{\bfseries \itshape #2,} \textcolor{mygray}{#3} \hfill {#4}
	
	
}
{}

%----------------------------------------
% poster abstract environment (printed)
%----------------------------------------
\newcommand{\poster}[3] %{title}{author}{affiliation}
{\filbreak %avoid page break
	
	{\bfseries \large #1} \\	
	\tab #2, \textit{#3}
	
}
{}

%----------------------------------------
% tags for talk type (colored circle in abstracts)
%----------------------------------------

\newcommand{\KLtag}{\tikz[baseline={([yshift=-.8ex]current bounding box.center)}]  \node[circle, inner sep=2pt, minimum size=0.5em, color=black, fill=\KLcolor]{\small \bfseries KL};} %colored circle with tag

\newcommand{\IStag}{\tikz[baseline={([yshift=-.8ex]current bounding box.center)}]  \node[circle, inner sep=2pt, minimum size=0.5em, color=black, fill=\IScolor]{\small \bfseries IS};} %colored circle with tag

\newcommand{\CTtag}{\tikz[baseline={([yshift=-.8ex]current bounding box.center)}]  \node[circle, inner sep=2pt, minimum size=0.5em, color=black, fill=\CTcolor]{\small \bfseries CT};} %colored circle with tag

\newcommand{\ITtag}{\tikz[baseline={([yshift=-.8ex]current bounding box.center)}]  \node[circle, inner sep=2pt, minimum size=0.5em, color=black, fill=\ITcolor]{\small \bfseries IT};} %colored circle with tag

%
% PAGE LAYOUT DEFINITIONS
%
\usepackage{etoolbox}

%------------------------------------------------------
% page style: vertical line on the side of each page
%------------------------------------------------------
\usepackage[scale=1,angle=0,opacity=1]{background}
\backgroundsetup{contents={}}

\AddEverypageHook{%
\ifthenelse{%
	\isodd{\thepage} \AND  \thepage>1 % if odd page but not front page
	}{%
	\backgroundsetup{
		color=\secondarycolor,
		position=current page.south east,%
		nodeanchor=south east,
		contents={\rule{10pt}{0.66\paperheight}}
		}
	}{%
	% nothing
	}
%
\ifthenelse{% 
	\NOT \isodd{\thepage} \AND \NOT \thepage=44% if even page
	}{%
	\backgroundsetup{
		color=\secondarycolor,
		position=current page.south west,%
		nodeanchor=south west,
		contents={\rule{10pt}{0.66\paperheight}}
		}
	}{%
	% nothing
	}
\BgMaterial}


%---------------------------------------------------
% chapter heading style
%---------------------------------------------------

\newdimen\mybarpadding
\mybarpadding=1.5em\relax %padding between gcolored bar and chapter name

\RedeclareSectionCommand[%
    ,afterskip=4em plus 1pt minus 1pt%
    ,beforeskip=-1pt%1.2em plus 1pt minus 1pt%
    ,level=0%
    ,toclevel=0%
]{chapter}%

\setkomafont{chapter}{\normalfont\normalsize\bfseries\Huge} % koma-script-specific command

\newcommand*{\mynumberedtest}[1]{% to test whether there is a number
  \if\relax\detokenize{#1}\relax%
  \else%
    #1%
    
  \fi}

%-------------------------------------------------chapter style definition

\renewcommand{\chapterlinesformat}[3]{%
  \ifthispageodd{%
    \hfill%
    \raisebox{-0.2em}{%
      \makebox[0pt][r]{\textcolor{\primarycolor}{\rule{\paperwidth}{1em}}}%
    }%
    \hspace{\mybarpadding}%
% 	\mynumberedtest{#2}
	\mbox{#3}%
  }{%
%    \hbox{%
%       \mynumberedtest{#2}
      \mbox{#3}%
      \hspace{\mybarpadding}%
      \raisebox{-0.2em}{%
        \makebox[0pt][l]{\textcolor{\primarycolor}{\rule{\paperwidth}{1em}}}%
      }%
%    }%
  }%
}
\makeatother
%---------------------------------------------------------

% TIMETABLE COLORS AND STYLES

% text and backgroud colors
\newcommand{\tbg}{gray} % background
\newcommand{\tfg}{white}
\newcommand{\tbc}{gray!25}

% talk types colors
\newcommand{\IScolor}{myblue!65} % invited speaker
\newcommand{\CTcolor}{white} % contributed talk
\newcommand{\KLcolor}{myorange!45} % keynote lecture
\newcommand{\ITcolor}{yellow!25} %

% row types
\newcommand{\tablebreak}[2]{% {time span}{break name}
	\rowcolor{\tbc} #1 &  \multicolumn{4}{c|}{\bfseries #2} \\ \hline }
\newcommand{\eventtype}[2]{% {time span}{event name}
	#1& \multicolumn{4}{c|}{\cellcolor{\tbg}\color{\tfg}\bfseries #2} \\ \hline }

% column spacing and position
\newcolumntype{L}[1]{%
	>{\raggedright\let\newline\\\arraybackslash\hspace{0pt}}m{#1}}
\newcolumntype{C}[1]{%
	>{\centering\let\newline\\\arraybackslash\hspace{0pt}}m{#1}}
\newcolumntype{R}[1]{%
	>{\raggedleft\let\newline\\\arraybackslash\hspace{0pt}}m{#1}}

%\newcommand{\mytable}{|C{0.15\linewidth}| C{0.05\linewidth}|  C{0.25\linewidth} C{0.1\linewidth} C{0.5\linewidth}|}

\newcommand{\IS}[5]{% {time span}{name}{University}{City, Country}{title}
	#1 &\cellcolor{\IScolor}IS&{\bfseries#2}\newline #4&&#5 \\ \hline}
\newcommand{\CT}[5]{%
	#1 &\cellcolor{\CTcolor}CT&{\bfseries#2}\newline #4&&#5 \\ \hline}
\newcommand{\KL}[5]{%
	#1 &\cellcolor{\KLcolor}KL&{\bfseries#2}\newline #4&&#5 \\ \hline}
\newcommand{\IT}[5]{%
	#1 &\cellcolor{\ITcolor}IT&{\bfseries#2}\newline #4&&#5 \\ \hline}
\newcommand{\tutorial}[5]{%
	#1 && {\bfseries#2}\newline #4 &&#5 \\ \hline}
