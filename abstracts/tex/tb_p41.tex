
    \begin{abstract_basarim}
    {Perşembe 16:00-16:15}
    {BASARIM2022}
    {Oda: 202}
    {Buse Yılmaz, Abdulkadir F. Yıldız}
    {A Graph Transformation Strategy for Optimizing SpTRSV}
    {%
    Buse YILMAZ, Abdulkadir Furkan YILDIZ}
    {%
    }
    {%
    Istinye University Dept. of Software Engineering - Istanbul, Turkey}
    Sparse triangular solve (SpTRSV) is an extensively studied computational kernel. An important obstacle in parallel SpTRSV implementations is that in some parts of a sparse matrix the computation is serial. By transforming the dependency graph, it is possible to increase the parallelism of the parts that lack it. In this work, we present an approach to increase the parallelism degree of a sparse matrix, discuss its limitations and possible improvements, and we compare it to a previous manual approach. The results provide several hints on how to craft a collection of strategies to transform a dependency graph. 
    
            \textbf{Index Terms} \newline{}Sparse Triangle Solve, spTRSV, sparse matrix, graph transformation, parallel computing
    \end{abstract_basarim}
    