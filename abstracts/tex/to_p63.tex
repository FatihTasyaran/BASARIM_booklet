
    \begin{abstract_online}{Kısmi Karışımlı bir Yanma Odası Tasarımında Girdap Etkilerinin İncelenmesi}{%
        Yunus Emre KARASU$^{1}$, Tuğba ZAMAN$^{1}$, Ayşe Gül GÜNGÖR$^{1}$, Dilay GÜLERYÜZ$^{2}$, Emre BÖNCÜ$^{2}$, Tahsin Berk KIYMAZ$^{2}$, Mehmet KARACA$^{3}$, Barış YILMAZ$^{4}$, Christophe ALLOUIS$^{5}$, İskender GÖKALP$^{5}$}{%
        }{%
        $^1$ İstanbul Teknik Üniversitesi Uçak ve Uzay Bilimleri Fakültesi - İstanbul, Türkiye\newline{}$^2$ Orta Doğu Teknik Üniversitesi Makina Mühendisliği Bölümü - Ankara, Türkiye\newline{}$^3$ Orta Doğu Teknik Üniversitesi Havacılık ve Uzay Mühendisliği Bölümü - Ankara, Türkiye\newline{}$^4$ Marmara Üniversitesi Makina Mühendisliği Bölümü - İstanbul, Türkiye\newline{}$^5$ Orta Doğu Teknik Üniversitesi Makina Mühendisliği Bölümü - Ankara, Türkiye}
    Bu çalışmada, kanatçıklı bir geometriye sahip, ön karışımsız bir yanma odasının soğuk akış analizleri üç farklı model için gerçekleştirilmiştir. Sayısal olarak kanatçıkları modellemek yüksek hesaplama maliyetlerini de beraberinde getirdiği için kanatçık geometrisi kullanılmadan akışa girdap etkisi kazandırmayı hedefleyen iki farklı model geliştirilmiştir. Üçüncü modelde yanma odası kanatçık ile birlikte modellenmiştir. Tüm modeller ile elde edilen sonuçlar deneysel veriler ile kıyaslanmıştır. \newline Girişte girdap etkisi kazandırılmış olan kanatçıksız model, eksenel ve teğetsel hızlarda deneye yakın sonuçlar elde etmiştir. Merkezi sirkülasyon bölgesinin yakalanmasında yalnızca kanatçık geometrisi dahil edilen model deneye yakınsamaktadır. 
    
            \textbf{Anahtar Kelimeler} \newline{}Girdap etkisi, büyük girdap benzeşimi, nümerik analiz, OpenFOAM
    \end{abstract_online}
    