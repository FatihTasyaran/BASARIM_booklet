
    \begin{abstract_basarim}
    {Çarşamba 11:15-11:30}
    {.}
    {Oda: 202 (fiziksel) - 111, 112 (çevrimiçi)}
    {Hüseyin Bahtiyar}
    {Örgü Kuantum Renk Dinamiğinde $\Omega_{c c}$ Baryonunun Yarı-Leptonik Geçişinin İncelenmesi}
    {%
    Hüseyin BAHTİYAR}
    {%
    }
    {%
    Mimar Sinan Güzel Sanatlar Üniversitesi Fizik Bölümü - İstanbul, Türkiye}
    Tılsımlı parçacıkların yarı leptonik bozunmalarını incelenmesi, standart modeli test etmede öne çıkan bir yöntemdir. Tılsımlı baryon sektörünün zayıf bozunmalarındaki son zamanlarda geliştirilen deneylerden gelen motivasyon ile $2+$ 1 çeşnili örgüler üzerinde $\Omega_{c c}^{+} \rightarrow \Omega_{c}^{0} \ell^{+} \nu$ geçişinin form faktörleri incelenmiştir. Öncelikte iki ve üç nokta fonksiyonları hesaplanmış, boyutsuz projeksiyonlu korelatörler çıkarılmış ve form faktörler oluşturulması için korelatörler birleştirilmiştir. Sıfir aktarılan momentum limitinde $f_{1}$ ve $g_{1}$ form faktörleri hesaplanmış diğer modellerle uyumlu olduğu gözlemlenmiştir. 
    
            \textbf{Anahtar Kelimeler} \newline{}Tılsımlı baryonlar, yarıleptonik form faktörler, Örgü KRD
    \end{abstract_basarim}
    