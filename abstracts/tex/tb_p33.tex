
    \begin{abstract_basarim}
    {Çarşamba 11:30-11:45}
    {BASARIM2022}
    {Oda: 202}
    {Serpil Y. Kuzu}
    {Study of Dilepton Spectrum with Machine Learning Approach at the LHC}
    {%
    Serpil Yalçın KUZU}
    {%
    }
    {%
    Firat University Faculty of Science, Department of Physics - Elazig, Turkey}
    Machine learning $(M L)$ application in diverse fields has been favored to elevate the performance of human endeavors due to its effectiveness in solving different types of real-world issues requiring intensive computation. In particle physics, the investigation of dimuons $\left(\mu^{+} \mu^{-}\right)$has an important role to understand various systems produced at high energy collisions. The absence of strong interaction makes these lepton pairs suitable to examine particles such as $J / \psi, \Upsilon$ and $Z$ boson. Dimuons can be identified by sophisticated techniques depending on cut-based methods as other particle analyses. In this contribution, the implementation of Random Forest and Weighted Random Forest ML models for defining dimuon pairs produced in proton-proton collisions at $7 \mathrm{TeV}$ at the LHC are discussed. 
    
            \textbf{Index Terms} \newline{}Dimuon, lepton pairs, particle analysis, random forest, weighted random forest, machine learning.
     \newline\newline\noindent \href{https://drive.google.com/file/d/1VTPtlXJdSI560csxqGXvoE4-9Lptr3pA/view?usp=drivesdk}{\bfseries Online Access}
    \end{abstract_basarim}
    