
        \begin{abstract}{Aerodinamik Veritabanı Modellerinin Adaptif Deney Tasarım Yöntemi ile İyileştirilmesi ve Sürecin Hızlandırılması}{%
            Ertan DEMİRAL}{%
            ROKETSAN A.Ş.}{%
            }
        Aerodinamik veritabanı modellerinin doğruluk seviyesi uçuş mekaniği çalışmalarının başarımı açısından kritik öneme sahiptir. Yüksek doğruluklu veritabanı modeli oluşturulması hem aerodinamik analiz yöntemine hem de analizlerin gerçekleştirileceği uçuş koşullarının doğru belirlenmesine bağlıdır. Bu çalışmada, aerodinamik veritabanı analizleri uçuş koşullarının adaptif deney tasarım yöntemi ile belirlenmesi amaçlanmıştır. Yapılan çalışmaların sonuçlarına göre, adaptif deney tasarım yöntemi örneklendirmeleri ile oluşturulan sürüklenme katsayısı modelinin doğruluk seviyesi statik deney tasarım yöntemi örneklendirmeleri ile oluşturulan modele göre iyileştirilmiştir. Bununla beraber, adaptif deney tasarım çalışmaları grafik işlemci birimi sistemleri üzerinde yürütülerek süreç hızlandırılmıştır. 
        \end{abstract}
        