
    \begin{abstract_online}{Dağıtık Ortamda Derece İlintileri Bazlı Gerçekçi Çizge Üretimi}{%
        Furkan ATAS, Mehmet Burak AKGÜN}{%
        }{%
        TOBB Ekonomi ve Teknoloji Üniversitesi Bilgisayar Mühendisliği - Ankara, Türkiye}
    Çizge üreteçleri, yeni geliştirilen topolojik algoritmaları test etmek için kullanılan araçlardır. Algoritmalar gerçek sistemler üzerinde çalıştırılmadan önce, yapay çizgeler üzerinde performansları değerlendirilir. Bu testlerin sonucuna göre algoritmada gerekli düzenlemeler yapılır ve tekrar test edilir. Bu döngü, algoritmanın performansı istekleri karşılayacak seviyeye gelinceye kadar devam eder. Fakat algoritmanın test edildiği yapay çizgenin ait olduğu alanın çizge özelliklerini taşıyıp taşımaması, test sonuçlarının gerçeğe yakınlığında doğrudan etkin bir role sahiptir. Örneğin, yeni tasarlanan bir ağ protokolünü test etmek için bir sosyal ağ kullanmak, araştırmacıların ağ protokolünün gerçek performansını görmesine engel olacaktır. Dolayısıyla yapay olarak üretilen çizgenin mümkün olduğunca gerçekçi olması gerekir. Rastgele üretilen bir çizgenin ait olduğu alana dair karakteristik özelliklerini korumak için gerçek bir çizgeyi kopyalamak en iyi yollardan biridir. Derece ilintileri bazlı yaklaşımlar çizgenin farklı derecelerdeki kenar-düğüm olasılıklarını kullanarak, verilen bir çizgeyi kopyalar. Günümüzde, internet erişiminin hem insanlar hem de cihazlar için kolaylaşmasıyla, üretilen veri hızla arttı. Büyük veri bir çok alanda kaçınılmaz hale geldi. Çizge verilerinin de büyük bir kısmı büyük veri seviyelerine geldi. Geleneksel tek makineli veri analitik çözümleri geniş ölçekli veriyi işlemek için yetersiz kaldı. Bu problemleri birleştirdiğimizde, büyük veri seviyesinde gerçekçi çizge üretimi problemi ortaya çıkmıştır. Bu çalışmada, Dk serisi kullanarak ölçeklenebilir algoritmaların tasarlanıo dağıtık olarak çalıştırılması gerçekleştirilmiştir. 
    
        \textbf{Keywords} \newline{}dağıtık hesaplama, yapay çizge üretimi, derece ilintileri, dk serisi
    \end{abstract_online}
    