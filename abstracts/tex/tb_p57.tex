\phantomsection

    \begin{abstract_basarim}
    {Perşembe 14:00-14:15}
    {BASARIM2022}
    {Oda: 202}
    {Amro A. Aljundi, Kamer Kaya}
    {Accelerating Graph Embedding on a Single GPU Using Graph Reordering}
    {%
    Amro Alabsi ALJUNDI, Kamer KAYA}
    {%
    }
    {%
    Sabanci University Faculty of Engineering and Natural Sciences \newline \noindent VERİM, Center of Excellence in Data Analytics - Istanbul, Turkey}
    Abstract-Graph embedding is the process of converting the irregular connectivity information of a graph into a structured, $d$-dimensional latent vector representation. This representation enables utilizing decades of existing machine learning models for graph-specific tasks such as link prediction and node classification. The excellent performance of graph embedding comes at the cost of memory and time, and the ever-increasing size of real-world graphs only exasperates these costs. Numerous algorithms accelerate graph embedding both on the CPU and on memory-limited GPUs. To alleviate the memory burden of graph embedding on GPUs, many of these algorithms will partition the parameter space of the problem into evenly sized parts that fit the GPU memory. In this work, we show that reordering graphs as a pre-processing step can substantially improve the speed of partition-based graph embedding algorithms. 
    
            \textbf{Index Terms} \newline{}Graph embedding, GPU programming, reordering, high-performance computing
     \newline\newline\noindent \href{https://drive.google.com/file/d/1mJKf5ZaYKOhfKoT0Dgo-n8cEwaoZl6r_/view?usp=drivesdk}{\bfseries Online Access}
    \end{abstract_basarim}
    