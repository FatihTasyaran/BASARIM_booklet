
    \begin{abstract_basarim}
    {Cuma 10:30-10:45}
    {BASARIM2022}
    {Oda: 202}
    {Ali C. Koyuncuoğlu, Çağrı Aydın, Uğur. O. Ünal, Barış Barlas}
    {Çekirdek Sayısının Hesaplamalı Gemi Direnci Analiz Performansına Etkileri}
    {%
    Ali Can KOYUNCUOĞLU, Çağrı AYDIN, Uğur Oral ÜNAL, Barış BARLAS}
    {%
    }
    {%
    İstanbul Teknik Üniversitesi Gemi İnşaatı ve Deniz Bilimleri Fakültesi Gemi İnşaatı ve Gemi Makineleri Mühendisliği - İstanbul, Türkiye}
    Bu çalışmada Hesaplamalı Akışkanlar Dinamiği vasıtasıyla bir açık deniz destek gemisi modelinin dalgalı su koşullarında iki serbestlik dereceli akış simülasyonları İstanbul Teknik Üniversitesi bünyesinde bulunan Ulusal Yüksek Başarımlı Hesaplama Merkezi imkanlarından faydalanılarak gerçekleştirilmiştir. Yüksek işlem gücü gerektiren çalışmada gemi modelinin ek dalga direncinin yanısıra farklı çekirdek sayılarının hesaplama sistemi performansına olan etkisi incelenmiştir. Hesaplamalar üç boyutlu, sıkıştırılamaz RANS denklemlerinin SST k-$\omega$ modeli ile birlikte çözümünü içermektedir. Serbest su yüzeyi VoF metodu kullanılarak modellenmiştir. Ağ örgüsü hücre sayısının performansa etkisinin anlaşılabilmesi maksadıyla iki farklı ağ yoğunluğu uygulanmıştır. Simülasyonlarda 128 çekirdek sayısına sahip 4 hesaplama sunucusu kullanılmıştır. Çekirdek sayısı sistematik olarak arttırılarak temel performans parametrelerindeki değişim gözlenmiştir. 
    
            \textbf{Anahtar Kelimeler} \newline{}gemi direnci, ek dalga direnci, HAD, çekirdek, paralel hesaplama
    \end{abstract_basarim}
    