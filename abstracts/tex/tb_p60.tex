
    \begin{abstract_basarim}
    {Çarşamba 14:30-14:45}
    {.}
    {Oda: 202 (fiziksel) - 104, 105, 111, 112 (çevrimiçi)}
    {Şener Özönder, H. Kübra Küçükkartal}
    {Yapay zeka ile hızlandırılmış nano-malzeme keşfi}
    {%
    Şener ÖZÖNDER$^{1}$, H. Kübra KÜÇÜKKARTAL$^{2}$}
    {%
    }
    {%
    $^1$ İstinye Üniversitesi Elektrik-Elektronik Mühendisliği - İstanbul, Türkiye\newline{}$^2$ Eskişehir Osmangazi Üniversitesi Bilgisayar Mühendisliği - Eskişehir, Türkiye}
    Kimyasallar, ilaçlar, biyomalzemeler ve alaşımlar gibi malzemelerin çalışılması uzun yıllara yayılan süreçler gerektirmektedir. Ayrıca bu malzemelerin bulunan fiziksel ve kimyasal özellikleri pratik uygulamalar için aranan özelliklerle aynı olmayabilir. Bu uzun süreç yeni yapay zeka ve optimizasyon yöntemleriyle tersine çevrilebilir. Bir malzemeyi ve onun yapısal olarak yakın türevlerini çalışmak yerine, o malzemenin tüm olası türevlerini içeren kimyasal veya yapısal parametre uzayı hızlı ve akıllı bir şekilde taranarak istenilen fiziksel veya kimyasal özelliğe sahip malzeme yapısı bulunabilir. Bu amaçla Bayes optimizasyonu, Gauss regresyonu ve yapay sinir ağları hem laboratuvarda sentez ile hem de bilgisayar ortamında simülasyonlar ile malzeme keşfini hızlandırmak için kullanılabilir. 
    
            \textbf{Anahtar Kelimeler} \newline{}malzeme keşfi, makine öğrenmesi, yapay zeka, Bayes optimizasyonu, Gauss regresyonu, yapay sinir ağları, grafen.
    \end{abstract_basarim}
    