
    \begin{abstract_basarim}
    {Cuma 11:30-11:45}
    {BASARIM2022}
    {Oda: 202}
    {Zuhal Altuntaş, Sanem Arslan, Betül Boz}
    {Approximate Execution of Critical Sections for Performance-Accuracy Tradeoff}
    {%
    Zuhal ALTUNTAŞ, Sanem ARSLAN, Betül BOZ}
    {%
    }
    {%
    Marmara University Computer Engineering Department - İstanbul, Turkey}
    Approximate computing enhances performance and energy efficiency of applications, while still achieving acceptable accuracy. Some of the multithreaded applications can tolerate the accuracy loss when critical sections are approximately executed, which in turn will eliminate the synchronization overhead of these applications and increase their performance. In this study, our objective is to explore the behavior of the critical sections and selectively skip the ones yielding performance improvements with an acceptable accuracy loss. We have observed the behaviour of 49 critical sections of 2 selected applications. Our experimental study indicates that skipping $76 \%$ of the critical sections offers 2.5x performance gain with $16 \%$ accuracy loss for Raytrace whereas $1.4 x$ performance improvement with $17 \%$ accuracy loss is obtained for Radiosity on the average when $36 \%$ of the critical sections are skipped. 
    
            \textbf{Index Terms} \newline{}approximate computing, multithreaded application, critical section, performance, accuracy
     \newline\newline\noindent \href{https://drive.google.com/file/d/1ZlPHw7wj9RfF-Qn_XKueCL8wqkUxazy5/view?usp=drivesdk}{\bfseries Online Access}
    \end{abstract_basarim}
    