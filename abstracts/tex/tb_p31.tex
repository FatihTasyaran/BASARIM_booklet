
    \begin{abstract_basarim}
    {Perşembe 10:30-10:45}
    {BASARIM2022}
    {Oda: 202}
    {Arash Karshenass, Özgür U. Baran, Onur Ata}
    {A Simplified Methodology For Rotor-Stator Interaction Problems With An Upwind Linearized Harmonic Solver}
    {%
    Arash KARSHENASS, Özgür Uğraş BARAN, Onur ATA}
    {%
    }
    {%
    Middle East Technical University Mechanical Engineering Department - Ankara, Turkey}
    The family of harmonic techniques for flow problems is an approach for solving unsteady but temporally periodic problems. These problems are often found in industrial applications such as turbomachinery, helicopter, aeromechanic, noise generation. Time-accurate solutions to such problems are computationally expensive. The harmonic methods are preferred in such problems as they benefit the temporal periodicity for solving the time-independent governing equations in the frequency domain. The linearized harmonic method (LH) is the less sophisticated member of the family of harmonic flow solvers. LH methods allow the mean and perturbed flowfield to be solved separately and in sequence. Simplifications lead to the LH method allow solutions with less computational cost than more advanced harmonic methods. In this study, we assess the performance of a new LH method in turbomachinery flows. For this purpose, inviscid, ideal gas is considered as the working fluid with a novel upwind flux discretization scheme. An explicit multistage Runge-Kutta method is applied for solving LH equations for compressible flow within a compressor stage where shocks are present within the channel. LH model is developed with Loci framework and coupled with a modified version of flowPsi open-source CFD solver. The test case involves the flowfield within the NASA rotor-stator37 stage. For simplicity, it is assumed that generated disturbance waves only affect the stator domain and hence, only unsteady flowfield in the stator is investigated. The model can be extended to take the rotor domain also into account. Comparing LH results to those from time-accurate shows that LH resulted in over-amplification of unsteady effect, but much sharper discontinuities can be captured compared to the time-accurate solver. 
    
            \textbf{Index Terms} \newline{}unsteady, turbomachine, stage37, rotor-stator, harmonic, linearized, Steger-Warming, upwind
     \newline\newline\noindent \href{https://drive.google.com/file/d/1-EnT5MmzhSr_BGdovjyG6qJsGFGORFBA/view?usp=drivesdk}{\bfseries Online Access}
    \end{abstract_basarim}
    