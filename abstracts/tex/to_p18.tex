
    \begin{abstract_online}{Tek Boyutlu, Sıralı, Büyük Ölçekli Veri Dizileri için Sıralı Örüntü Madenciliği Yaklaşımı}{%
        Ali Burak CAN$^{1}$, Meryem UZUN-PER$^{1,2}$, Mehmet S. AKTAŞ$^{3}$}{%
        }{%
        $^1$ BiletBank Research and Development Center, Akdeniz PE-TUR A.Ş - Istanbul, Turkey\newline{}$^2$ Istanbul Health and Technology University Computer Engineering Department- Istanbul, Turkey\newline{}$^3$ Computer Engineering Department, Yildiz Technical University - Istanbul, TurkeyYildiz Technical University Computer Engineering Department - Istanbul, Turkey}
    Sıralı örüntü madenciliği algoritmaları, belirli bir sıraya dayalı olarak bir araya gelmiş sıralandırılmış veri dizileri (içinde bir ya da birden fazla eleman bulunan veri dizileri - 1-sequence, n-sequence) üzerinde, sıralı örüntülerin bulunmasını sağlayan gözetimsiz makine öğrenmesi algoritmalarıdır. Literatürde, bu kategoride yer alan algoritmaları incelediğimizde; bu algoritmaların, uzunluğu birden fazla olan sıralı veri dizileri (n-sequences) için optimize edildikleri görülmektedir. Buradan yola çıkarak, genom dizisi verileri gibi, tek eleman içeren (1-sequence) sıralı veri dizilerinden oluşan veri setlerine yönelik optimize edilmiş sıralı örüntü tespiti algoritmalarına ihtiyaç olduğu görülmektedir. Bu araştırma kapsamında, tek elemanlı sıralı veri dizileri (1-sequence) içeren veri setleri üzerinde, sıralı örüntüleri yüksek performanslı bir şekilde tespit edilebilecek bir algoritmanın tasarlanması ve geliştirilmesi problemi üzerinde çalışılmaktadır. Yine bu araştırma kapsamında, tek bir bilgisayarın veri saklama ortamlarında (bellek ve fiziksel disk) tutulamayacak büyüklükte olan veri setleri üzerinde, sıralı örüntülerin tespitine olanak verecek bir sıralı örüntü madenciliği algoritması üzerinde çalışılmaktadır. Önerilen algoritmanın çalışırken ihtiyaç duyduğu zaman ve bellek gereksinimleri deneysel olarak irdelenmiştir. Elde edilen sonuçlar, önerilen algoritmanın literatürde yer alan benzer kategorideki algoritmalarla, aynı doğruluk derecesine ulaşırken, daha az çalışma süresine sahip olduğunu göstermektedir. Elde edilen sonuçlar ümit vericidir. 
    
            \textbf{Anahtar Kelimeler} \newline{}Sıralı Örüntü Madenciliği, GSP, PrefixSpan, Spark
    \end{abstract_online}
    