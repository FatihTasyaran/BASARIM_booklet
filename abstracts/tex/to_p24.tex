
    \begin{abstract_online}{Paralel İşlerin Çizelgelemesinde İşlemci Tahsisi için Hipersezgisel Yaklaşım}{%
        Gülçin BOLAT, Fatih Erdoğan SEVİLGEN}{%
        }{%
        Gebze Teknik Üniversitesi Bilgisayar Mühendisliği Bölümü - Kocaeli, Türkiye}
    Paralel bilimsel uygulamaların çizelgelenmesi kaynakların verimli kullanılması ve işlemlerin tamamlanma süresinin azaltılması için oldukça önemlidir. Paralel uygulamalar veri paralelliği ve işlem paralelliği içerir. Çoklu işlemcili sistemlerde sadece işlem paralelliği içeren uygulamaların çizelgelenmesi probleminin NP-Tam problem olduğu bilinmektedir. Veri paralel işlemlerinde eklenmesi ile problem daha da zorlaşmaktadır. Yapılan çalışmalarda problemin işlemci tahsisi ve çizelgeleme aşaması için çeşitli algoritmalar önerilmiştir. Bu çalışmada işlemci tahsisi süreci için çeşitli sezgisel algoritmalar ve genetik algoritma birlikte kullanılarak özgün bir hipersezgisel yaklaşım sunulmaktadır. Deneysel sonuçlar, algoritmanın açgözlü algoritmalar ile karşılaştırıldığında daha iyi performans sonuçları sunduğunu göstermektedir. 
    
        \textbf{Keywords} \newline{}Statik Çizelgeleme, Veri Paralelliği, İşlemci Tahsisi, Genetik Algoritma
    \end{abstract_online}
    