
    \begin{abstract_basarim}
    {Perşembe 11:45-12:00}
    {BASARIM2022}
    {Oda: 202}
    {Ali Can Fadıl, Baha Zafer}
    {Parallel Aeroacoustic Computation of Unsteady Transonic Cavity Flow via Open CFD Source Codes}
    {%
    Ali Can FADIL, Baha ZAFER}
    {%
    }
    {%
    İstanbul Technical University Department of Aeronautical Engineering and Astronautical Engineering - İstanbul, Turkey}
    Cavity flow research has been ongoing experimentally since the $1940 \mathrm{~s}$, especially for weapon bay use in fighter aircraft. With the development of technology, experimental studies have begun to be simulated. With the emergence of High-Performance Clusters, the success of these simulations has increased and simulation studies have begun to replace experimental studies. In this paper, an open rectangular, unsteady transonic cavity with a length to depth ratio of 5 , Mach number $0.85$ and Reynolds number of approximately $6.5 \times 10^{6}$ was simulated using High-Performance Cluster. Likewise cavity doors were used to model a real weapon bay. Detached Eddy Simulation was used to resolve turbulent properties in the flow domain. Results compatible with experimental results were obtained with OpenFOAM® , an open-source CFD code based on the finite volume method. 
    
        \textbf{Keywords} \newline{}Cavity Flow, OpenFOAM, Aeroacoustics, Parallel CFD
    \end{abstract_basarim}
    