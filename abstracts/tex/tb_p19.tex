\phantomsection

    \begin{abstract_basarim}
    {Çarşamba 14:15-14:30}
    {BASARIM2022}
    {Oda: 202}
    {Gamze Atalmış, Serkan Toros, Nebi Yelegen, Yüksel Kaplan}
    {Effect of Copper-Coated Storage Materials on Reaction Kinetics}
    {%
    Gamze ATALMIŞ$^{1}$, Serkan TOROS$^{2}$, Nebi YELEGEN$^{1}$, Yüksel KAPLAN$^{1}$}
    {%
    }
    {%
    $^1$ Niğde Ömer Halisdemir University Mechanical Engineering Department, Prof. Dr. T. Nejat Veziroglu Clean Energy Research Center - Nigde, Turkey\newline{}$^2$ Niğde Ömer Halisdemir University - Nigde, Turkey}
    In this study, hydrogen storage in metal hydride reactors was investigated numerically. A mathematical model including complex heat and mass transfer, which considers the flow occurring during the hydrogen charge/discharge process in metal hydride reactors, has been developed. In the experimental study, the thermal conductivity of the storage material, which was coated with copper and turned into pellets, was improved by 500-750 percent in order to accelerate the hydrogen charge/discharge processes and to get the needed hydrogen in a short time and at the desired flow rates. The developed macro modeling was solved numerically with the help of the COMSOL Multiphysics® software package. A two-dimensional axisymmetric model was developed to study the hydrogen absorption reaction. Simulation studies have shown that permeability of the metal hydride to be used in the study and thermal conductivity are essential for the optimization design of the metal hydride tank. 
    
            \textbf{Index Terms} \newline{}hydrogen storage, metal hydride, two-dimensional axisymmetric.
     \newline\newline\noindent \href{https://drive.google.com/file/d/1PZiPStwp-8dYWlP1FF3Pwcvxqa3uXY1x/view?usp=drivesdk}{\bfseries Online Access}
    \end{abstract_basarim}
    