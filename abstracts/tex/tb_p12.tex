
    \begin{abstract_basarim}
    {Cuma 9:30-9:45}
    {BASARIM2022}
    {Oda: 202}
    {Ömer Tayfuroğlu, Abdulkadir Koçak, Yunus Zorlu}
    {Development of a Generic Neural Network Potential for IR-MOF Series}
    {%
    Ömer TAYFUROğLU, Abdulkadir KOÇAK, Yunus ZORLU}
    {%
    }
    {%
    Gebze Technical University Department of Chemistry - Gebze, Kocaeli, Turkey}
    Abstract-Metalorganic frameworks (MOFs) with their exceptional porous and organized structures have been subject of numerous applications. Predicting macroscopic properties from atomistic simulations require the most accurate force fields, which is still a major problem due to MOFs' hybrid structures governed by covalent, ionic and dispersion forces. Application of ab-initio molecular dynamics to such large periodic systems are thus beyond the current computational power. Therefore, alternative strategies must be developed to reduce computational cost without losing reliability. In this work, we describe the construction of a generic neural network potential (NNP) for IRMOFn series $(n=1,4,7,10)$ trained by PBE-D4/def2-TZVP reference data of MOF fragments. We validated the resulting NNP on both fragments and bulk MOF structures by prediction of properties such as equilibrium lattice constants, phonon density of states and linker orientation. The energy and force RMSE values for the fragments are only $0.0017$ $e V / a t o m$ and $0.15 \mathrm{eV} / \AA$, respectively. The NNP predicted equilibrium lattice constants of bulk structures, which are not included in training, are off by only 0.2-2.4\% from experimental results. Moreover, our fragment trained NNP greatly predicts phenylene ring torsional energy barrier, equilibrium bond distances and vibrational density of states of bulk MOFs. The publicly available pre-trained model opens the door to investigate different aspects of IRMOFs at the first principle level accuracy. 
    
            \textbf{Index Terms} \newline{}Machine learning, Metal-organic framework, Neutral network potential, DFT
    \end{abstract_basarim}
    