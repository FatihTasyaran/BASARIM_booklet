
    \begin{abstract_basarim}
    {Çarşamba 14:00-14:15}
    {BASARIM2022}
    {Oda: 202}
    {Ayas Kiser, Yeliz Gürdal}
    {Theoretical Insight Into Hydrogen Evolution Mechanism of Pentadentate Molecular Catalyst Having Cobalt As Reaction Center}
    {%
    Ayas KİSER, Yeliz GÜRDAL}
    {%
    }
    {%
    Adana Alparslan Türkeş Science and Technology University Bioengineering Department - Adana, Turkey}
    Hydrogen $\left(\mathrm{H}_{2}\right)$ molecule is a significant carrier of energy that is found naturally in biomass, water, and hydrocarbon. In this study, by applying Abinitio Molecular Dynamics simulations, we theoretically investigate water reduction mechanism of Co-based poly-pyridyl catalyst having octahedral coordination. For the hydrogen production cycle, ECEC mechanism is utilized and each intermediate step is simulated in order to determine the allowed spin states as well as solvent response around the reaction center. We demonstrate that, following the first electron injection, it is more likely for the catalyst to continue to the first protonation intermediate step with singlet multiplicity. Subsequently, following the first protonation step, Co center gains the required proton from the water molecule in the first solvation shell. Along the simulation, fast rotation of the nearest water molecule certifies the efficiency of the $\mathrm{CoPy}_{5}$ for the water reduction reaction. 
    
            \textbf{Index Terms} \newline{}Ab-initio molecular dynamics, $\mathrm{H}_{2}$ production, water reduction
    \end{abstract_basarim}
    