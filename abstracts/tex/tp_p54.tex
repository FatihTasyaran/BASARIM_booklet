
        \begin{abstract}{Static and Dynamic Modeling of N-Methyl-Indole \newline\noindent (N=1-6) in Water at the B3LYP/AMBER Level Using the COBRAMM Interface}{%
            Caglar KARACA}{%
            Manisa Celal Bayar University}{%
            }
        Computational dynamic emission spectroscopy in rigid medium solvents is a highly difficult technique. In recent years, technological improvement makes realistic models capable of experimental observables is possible. In this study, we have improved a successful simulation strategy in excitation and emission energies using hybrid models. The selected high layer target molecules and low and mobile layer water molecules are optimized with together. The hybrid QM/MM level is a powerful tool to efficiently is described the interactions of a molecule with its solvent medium. In this context, we simulate static and dynamic excited and emission spectra using COBRAMM interface protocol at the B3LYP/AMBER for rigid solvent models, TIP3P models are used within the mobile MM layer up to 50 nanometers radius away, for methyl derivatives of indole to a room-temperature. The QM/MM optimization calculations give us reliable structures both ground and excited states. Energy fluctuation of systems involves four states starting on the S0-S3, computations have been carried out by the same level for 150 femtoseconds. These calculated processes in ultrafast time scale have been explained how the evolution of excited and emission spectra when solvent molecules are movable. S0 and S1 geometry optimization of molecule in water droplet consisting of 500 TIP3P water are computed B3LYP/6-311++G(d,p) basis set and all low and mobile layer data was obtained Amber GAFF force field. S1 state geometry is converged approximately within 120 optimization cycles while the S0 optimization cycle takes longer time because of librational movements of water. Because this librational motion causes chaos at the RMS/D value, the number of mobile molecules has been reduced from the optimization steps. The energy difference between the first excited and ground state has a fluctuating character. The distribution of these fluctuations has been analyzed to create Fluctuating Gap Distribution (FGD) which reveals the most appropriate excitation/emission wavelengths. 
        \end{abstract}
        