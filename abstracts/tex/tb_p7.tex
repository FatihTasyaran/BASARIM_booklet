
    \begin{abstract_basarim}
    {Perşembe 11:00-11:15}
    {BASARIM2022}
    {Oda: 202}
    {Tamer Şener, Burakhan Şüküroğlu, Ayşe G. Güngör}
    {Kimyasal tepkimeli türbülanslı akışlar için hesaplamalı akışkanlar dinamiği çözücüsü, lestr3d}
    {%
    Tamer ŞENER, Burakhan ŞÜKÜROĞLU, Ayse G. GUNGOR}
    {%
    }
    {%
    İstanbul Teknik Üniversitesi Uçak ve Uzay Bilimleri Fakültesi - İstanbul, Türkiye}
    Tepkime sonucunda oluşacak türler ile taze, yanmamış gazların karışabilmesini sağlamak üzere yanma odalarında genellikle türbülanslı akış tercih edilir. Bu gibi problemlerde, kimyasal tepkime ile türbülansın etkileşimi sebebiyle ortaya çıkan akış yapılarının zaman ve uzay ölçekleri oldukça küçülür ve bu yapıların zengin fiziğini modelleyebilmek üzere yüksek başarımlı hesaplama ihtiyacı doğar. Kullanılan yazılımın ise yüksek başarımlı hesaplama platformlarında kullanılan çekirdek sayısı artışı ile hesaplama hızında artış gözlemlenmeli ve bu yazılım platformdan bağımsız olmalıdır. Bu sebeple bu çalışmada, türbülanslı yanma benzetimleri için geliştirilen paralel akış çözücüsü lestr3d, yüksek başarımlı platformlarda test edilmiş ve bu yazılımın ölçeklenebilir olduğu gösterilmiştir. Ayrıca, çekirdek sayısı ile hızlanma performansı tür deklemlerinin çözülmesi ve çözülmemesi durumunda test edilmiştir. Bununla beraber bu yazılım farklı platformlarda, UHeM ve TRUBA, çalıştırılarak aynı sonucu verdiği ve platformdan bağımsız olduğu gösterilmiştir. Bunlara ek olarak akış çözücüsü lestr3d’nin türbülanslı küt bir cisim etrafındaki akış benzetimi sonuçları deneysel sonuçlar ile kıyaslanmış ve bu sonuçların birbiri ile uyumlu olduğu gösterilmiştir. 
    
            \textbf{Anahtar Kelimeler} \newline{}Türbülanslı akışlar, Yanma, Yüksek Başarımlı Hesaplama, Paralel Programlama, Büyük Girdap Benzetimi
    \end{abstract_basarim}
    