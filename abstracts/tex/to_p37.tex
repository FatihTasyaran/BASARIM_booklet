
    \begin{abstract_online}{Size dependent change of mean square displacement in gold nanocrystals: \newline A molecular dynamics simulation}{%
        Merdan BATYROW$^{1}$, İlknur ERUÇAR$^{2}$, Hande ÖZTÜRK$^{1}$}{%
        }{%
        $^1$ Özyeğin University Dept. of Mechanical Engineering - İstanbul, Turkey\newline{}$^2$ Özyeğin University Dept. of Natural and Mathematical Sciences - İstanbul, Turkey}
    Mean square displacements (MSDs) of 5, 10, 15, 20, and 30 nm diameter spherical gold nanocrystals were studied with Molecular Dynamics (MD) simulations at room temperature and below. Our analyses show that there is a strong size and temperature dependency of the MSD of spherical gold nanocrystals. Moreover, these displacements increase radially from the center of the nanocrystals and reach a maximum at the surface layers due to the presence of undercoordinated surface atoms and their relatively unrestricted motions. Results of this work will be useful to understand the effect of nanocrystal size on quantifying the amplitude of atomic vibrations and their effects on measured intensities from their x-ray diffraction data. 
    
            \textbf{Index Terms} \newline{}Molecular dynamics simulations, gold nanocrystal, mean square displacement, x-ray diffraction, Debye-Waller factor.
    \end{abstract_online}
    