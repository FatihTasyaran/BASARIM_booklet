
        \begin{abstract}{Makine Öğrenmesi tabanlı Gerçek Zamanlı Hedef Tespiti için Güç Verimli Paralel Hesaplama}{%
            Alparslan FİŞNE}{%
            ASELSAN A.Ş.}{%
            }
        Bu çalışmada, eşik tabanlı Sabit Yanlış Alarm Oranı (SYAO) radar hedef tespit yöntemine göre daha yüksek doğruluk sunan makine öğrenmesi tabanlı radar hedef tespiti için güç-verimli ve gerçek zamanlı hesaplama tasarımları anlatılmaktadır. SYAO yöntemine benzer şekilde iki boyutlu (2B) tarama yaparak hedef varlığını kontrol eden makine öğrenmesi uygulaması, Evrişimsel Sinir Ağı (Convolutional Neural Network, CNN) modeli içermektedir. CNN modelinin taramalı hesaplama yapısından dolayı tekraren veriler işlemeye alındığı için hesaplama kaybı gerçekleşmektedir. Önerilen katman optimizasyonları sayesinde taramalı yöntem kaldırılarak ve paralel hesaplama yapılarak gerçek zamanlı hesaplama sağlanmıştır. Güç-verimli bir hesaplama mimarisi için Jetson AGX Xavier GPGPU birimiyle 15 W ve 30 W güç modlarında Intel Xeon W-2123 CPU biriminden sırasıyla 1.2x ve 2x hızlanma sağlanmış olup düşük güç tüketimi ile gerçek zamanlı tespit uygulaması başarıyla gerçekleştirilmiştir. 
        \end{abstract}
        