
    \begin{abstract_basarim}
    {Çarşamba 11:45-12:00}
    {.}
    {Oda: 202 (fiziksel) - 104, 105, 111, 112 (çevrimiçi)}
    {İlkay T. Çakır, Sajad Einy, Sinan Kuday}
    {Preliminary Data Recognition on Collision Images at Large Hadron Collider with Machine Learning Techniques}
    {%
    İlkay Türk ÇAKIR$^{1,2}$, Sajad EINY$^{3}$, Sinan KUDAY$^{4}$}
    {%
    }
    {%
    $^1$ Ankara University Faculty of Engineering, Department of Artifical Intelligence Data Engineering - Ankara, Turkey\newline{}$^2$ Giresun University Faculty of Engineering, Department of Energy Systems Engineering - Giresun, Turkey\newline{}$^3$ İstanbul Aydin University, Faculty of Engineering, Department of Computer Engineering - İstanbul, Turkey\newline{}$^4$ Ankara University, Faculty of Science, Department of Physics - Ankara, Turkey}
    A preliminary analysis have been implemented to distinguish good events in Monte Carlo (MC) data using image recognition tools. All images from MC simulations are created with DELPHES detector program with ATLAS detector card. The experimental dataset contains some of the signal events in three different forms: side front, and ηφ-plane. We increased the number of images to 600 with improving image resolution, and quality by utilizing Image Data Augmentation code from Keras (Python deep learning API). Furthermore, we utilized transfer learning from ImageNet database due to overfitting problems. Different types of well-known models are employed for classification task such as VGG16, VGG19, Resnet 50, MobileNetv2 and Xception.  
    
            \textbf{Index Terms} \newline{}ATLAS, machine learning, deep learning algorithm, Delphes
    \end{abstract_basarim}
    