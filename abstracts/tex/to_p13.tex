
    \begin{abstract_online}{İzometrik Kasılmada Aponevroz Örtüsünün Rolü ve Şeklinin Eniyileme Yöntemiyle Tahminine Yönelik Hesaplamalar}{%
        Şükrü Furkan TAŞDEMİR, Cevat Volkan KARADAĞ, Ali Fethi OKYAR}{%
        }{%
        Yeditepe Üniversitesi Mühendislik Fakültesi Makine Mühendisliği Bölümü - İstanbul, Turkey}
    Bu çalışmada sayısal olarak modellenmiş kurbağa gastrocnemius (plantaris longus) kasının sonlu elemanlar yöntemiyle yapılan kasılma benzetimi (simülasyon) ile üzerinde taşıdığı ince bir zar olan aponevroz örtüsünün geometrisin bulunmasına çalışılmıştır. Her bir örnek için çözüm üretilmesi standart bir ofis bilgisayarında ortalama 200 saniyelik işlem süresi gerektirmektedir. Örtü geometrisinin tanımlanmasında gereken parametrelerin fazlalığı, çok yüksek sayıda yerel minimum değerlerinin elimine edilmesi gerekliliği ve belirli bir karar uzayında en iyinin bulunması gerektiğinden genetik algoritma seçilmiştir. Genetik algoritma kullanılarak yapılması hedeflenen örtü geometrisi eniyileme süreci için yaklaşık 20 bin işlemci-saat ihtiyacı başgöstermektedir. Bu ihtiyaç göz önüne alınarak Ulusal Hesaplama Merkezi’ne (UHEM) proje başvurusunda bulunulmuş ve çalışmanın devamı orada yapılmıştır. UHEM’de MATLAB-FEAP paralel işlem senaryosu oluşturulmuştur. Oluşturulan bu senaryo ile öncelikle parametre uzayının taranması ve ardından aponevroz örtü geometrisinin bulunması çalışması başarı ile tamamlanmıştır. Sonuç olarak, elde edilen aponevroz örtüsünün fizyolojik ve mekanik bulguları irdelenmiştir. 
    
    \end{abstract_online}
    