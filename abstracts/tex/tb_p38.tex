
    \begin{abstract_basarim}
    {Thursday 11:15-11:30}
    {Modeling and Simulation}
    {G039}
    {RANS and LES of a Turbulent Non-Premixed Flame Using OpenFOAM}
    {%
    Yunus Emre KARASU, Tugba ZAMAN, Ayse G. GUNGOR}
    {%
    }
    {%
    İstanbul Technical University Faculty of Aeronautics and Astronautics - Istanbul, Turkey}
    This paper presents numerical simulations of non-premixed methane-air diffusion flame, Sandia Flame D. The numerical simulations are performed with OpenFOAM. Reynolds averaged Navier–Stokes (RANS) approach is employed with different combustion and radiation models, and chemistry mech- anisms. The results are compared with each other and against experimental data. It is found that the partially stirred reactor (PaSR) combustion model gives accurate results considering its computational cost and numerical accuracy with reduced GRI-Mech 3.0 and P1 radiation heat transfer model. RANS results are in reasonable agreement with experimental data and computationally affordable. Subsequently, large eddy simulation (LES) approach is employed. LES gives very good predictions of temperature profiles and species mixing. Although LES approach comes with increased CPU cost, it is required for simulating time-dependent flow characteristics and dynamics of the flame alongside enhanced numerical accuracy of mean profiles. 
    
        \textbf{Keywords} \newline{}Turbulent combustion, methane-air flame, RANS, LES, high performance computing.
    \end{abstract_basarim}
    