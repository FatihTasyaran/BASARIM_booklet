
        \begin{abstract}{GPU-accelerated floating random walk based transient heat conduction solver}{%
            Ersin YILDIZ}{%
            Özyeğin University}{%
            }
        Solving diffusion processes with random walk Monte Carlo methods offers certain advantages over the discretization-based methods. One of the main advantages of these types of methods is that they can solve for the temperature at only selected points as opposed to the other discretization-based methods that require solving for the entire physical domain even though the temperature at a single physical point is of interest. In this study, a floating random walk (FRW) method, a type of meshless random walk method, has been implemented on graphics processors. In this method, the integral representing the analytical solution at a local point is evaluated by the statistical sampling from the possible random paths between the point and the domain boundaries. The implementation has been verified on two test problems with analytical solutions. The standard deviations in the results were quantified running the test problems with the increasing number of particles. Except for a deficient number of particles, the GPU-accelerated solver showed a clear speed advantage over the serial code running on the CPU. Running the solver with 105 particles resulted in an x100 speed-up on a gaming-level graphics card. Finally, the implemented solver was also tested on a more realistic transient heat conduction problem and the results were compared with laplacianFoam, an implicit finite volume based transient heat conduction solver available in OpenFOAM. It was found that the GPU-accelerated solver with 106 particles runs about x10 faster in comparison to laplacianFoam running in serial on CPU. It is anticipated that a further increase in the speed-up could be reached by parallelizing the algorithm on multiple graphics processors. 
        \end{abstract}
        