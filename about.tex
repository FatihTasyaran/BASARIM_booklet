\section{BAŞARIM Konferansları ve BAŞARIM 2022}

Yüksek Başarımlı Hesaplama (YBH), bir süperbilgisayar üzerindeki milyonlarca işlemciyi kullanarak tek bir işlemci ile günler sürebilecek bir hesaplamayı bir dakikadan az bir sürede gerçekleştirebilir. Verinin büyük, karmaşık ve farklı kaynaklardan geldiği bütün alanlarla iç içe çalışır ve çığır açabilecek sonuçlara daha az masrafla, daha kısa sürede ve daha verimli bir şekilde erişmenizi sağlar. Ayrıca YBH, bir kalp üzerinde yapılacak operasyonların yan etkilerini operasyonları gerçekleştirmeden, uçağınız için en verimli kanatları o kanatları üretmeden, ya da bir tedavinin bir hasta üzerindeki etkisini tedaviyi uygulamadan size gösterebilir.
Ülkemizde YBH alanında yapılan çalışmalar ve sonuçları belirli bir olgunluğa erişmiş ve bu konuya ilişkin olarak yapılan atılımlar hız kazanmıştır.
İlk akla gelen örnekler arasında İTÜ bünyesinde hizmete girmiş olan Ulusal Yüksek Başarımlı Hesaplama Merkezi,
ileri aşamalarına ulaşmış olan TRUBA Türk Ulusal e-Bilim e-Altyapısı (eski ismi ile TR-Grid Ulusal Grid Oluşumu) ve yürütülen ulusal/uluslararası ölçekteki projeler verilebilir.

BAŞARIM konferansları serisi, TR-Grid Ulusal Grid Oluşumu'nun çalışmaları ve bütünleştirici etkisi altında ortaya çıkmıştır ve
YBH alanında gerçekleştirilen çalışmaların yaygınlaştırılmasını, elde edilen sonuçların ve deneyimlerin, ulusal ölçekte süreklilik gösteren
bir konferans kapsamında değerlendirilmesini ve paylaşılmasını, ve bu teknolojiyi kullanan paydaşların bir araya getirilmesini
hedeflemektedir. Bu hedefler doğrultusunda, ilki 2009'da, sonunucusu 2020'de olmak üzere, bugüne kadar toplam altı BAŞARIM konferansı düzenlenmiştir.

\begin{itemize}
    \setlength{\itemsep}{0pt}
  \setlength{\parskip}{0pt}
\item \href{http://basarim09.ceng.metu.edu.tr/}{1. Ulusal Yüksek Başarımlı ve Grid Hesaplama Konferansı}, 15-18 Nisan 2009, Orta Doğu Teknik Üniversitesi, KKM, Ankara
\item \href{https://www.basarim.org.tr/2010/}{2. Ulusal Yüksek Başarımlı ve Grid Hesaplama Konferansı}, 10-13 Temmuz 2010, İstanbul Teknik Üniversitesi, SDKM, İstanbul
\item \href{https://www.basarim.org.tr/2012/}{3. Ulusal Yüksek Başarımlı Hesaplama Konferansı}, 12-13 Nisan 2012, Bilkent Üniversitesi, Ankara
\item \href{https://www.basarim.org.tr/2015/}{4. Ulusal Yüksek Başarımlı Hesaplama Konferansı}, 1-2 Ekim 2015, Kültür ve Kongre Merkezi, ODTÜ, Ankara
\item \href{https://www.basarim.org.tr/2017/}{5. Ulusal Yüksek Başarımlı Hesaplama Konferansı}, 14-15 Eylül 2017, Yıldız Teknik Üniversitesi, Davutpaşa, İstanbul
\item \href{https://www.basarim.org.tr/2020/}{6. Ulusal Yüksek Başarımlı Hesaplama Konferansı}, 8-9 Ekim 2020, Ankara Yıldırım Beyazıt Üniversitesi, Ankara
\end{itemize}

Bu konferansların yedincisi BAŞARIM 2022 başlığı altında, TÜBİTAK ULAKBİM, Sabancı Üniversitesi, Orta Doğu Teknik Üniversitesi iş birliği
ve EuroCC@Türkiye projesinin desteği ile 11-13 Mayıs 2022 tarihleri arasında Sabancı Üniversitesi, Altunizade Dijital Kampüs'te, İstanbul'da hibrit bir yapıda (çevrimiçi/yüzyüze) düzenlenmektedir. 

\section{Düzenleme Komitesi}
\begin{tabular}{ll}
Bekir Bediz & Sabancı Üniversitesi \\
Ahmet Demirelli & Sabancı Üniversitesi \\
Merve Demirtaş & TÜBİTAK ULAKBİM \\
Özcan Dülger & Orta Doğu Teknik Üniversitesi \\
Mehmet Karaca & Orta Doğu Teknik Üniversitesi \\
Pınar Karagöz & Orta Doğu Teknik Üniversitesi \\
Kamer Kaya & Sabancı Üniversitesi \\
Murat Manguoğlu & Orta Doğu Teknik Üniversitesi \\
Öznur Taştan & Sabancı Üniversitesi \\
Hande Toffoli & Orta Doğu Teknik Üniversitesi \\
Özlem Sarı & TÜBİTAK ULAKBİM \\
Sevil Sarıkurt & TÜBİTAK ULAKBİM \\
Aras Saygın & Orta Doğu Teknik Üniversitesi \\
Cevat Şener & Orta Doğu Teknik Üniversitesi \\ 
Nilay Sezer Uzol & Orta Doğu Teknik Üniversitesi \\
Hüsnü Yenigün & Sabancı Üniversitesi \\
Sinan Kaan Yerli & Orta Doğu Teknik Üniversitesi \\
\end{tabular}

\section{Yerel Organizasyon Komitesi}
\begin{tabular}{ll}
Amro Alabsi Aljundi & Sabancı Üniversitesi \\
Başak Amasya & Sabancı Üniversitesi \\
Kaya Gökalp & Sabancı Üniversitesi \\
Şeyma Selcan Mağara & Sabancı Üniversitesi \\
Arda Şener & Sabancı Üniversitesi \\
Fatih Taşyaran & Sabancı Üniversitesi \\
Ahmet Yazıcı & Sabancı Üniversitesi \\
Ceren Yıldırım & Sabancı Üniversitesi
\end{tabular}

\section{Değerlendirme Komitesi}
\begin{tabular}{ll}
Adnan Özsoy & Hacettepe Üniversitesi \\
Ahmet Ercan Topçu & American University of the Middle East \\
Ahmet Erdem Sarıyüce & University of Buffalo \\
Ahmet Fatih Mustaçoğlu & TÜBİTAK BİLGEM \\
Ahmet Sayar & Kocaeli Üniversitesi  \\
Antoine Marion & Orta Doğu Teknik Üniversitesi \\
Barış Aktemur & Intel GmbH \\
Barış Ethem Süzek & Muğla Sıtkı Koçman Üniversitesi \\
Bora Uçar & ENS Lyon \\
Bülent Çatay & Sabancı Üniversitesi \\
Buse Yılmaz & İstinye Üniversitesi \\
Can Özturan & Boğaziçi Üniversitesi \\
Canan Atilgan & Sabancı Üniversitesi \\
Cem Sevik & Eskişehir Teknik Üniversitesi \\
Didem Unat & Koç Üniversitesi \\
Emre Sururi Taşçı & Hacettepe Üniversitesi \\
Enver Özdemir & İstanbul Teknik Üniversitesi \\
Erol Şahin & Orta Doğu Teknik Üniversitesi \\
Erol Yıldırım & Orta Doğu Teknik Üniversitesi \\
Ferhat Özgür Çatak & Simula Research Laboratuary \\
Galip Aydın & Fırat Üniversitesi \\
Gürhan Gündüz & Muğla Sıtkı Koçman Üniversitesi \\
Harun Koku & Orta Doğu Teknik Üniversitesi \\
Hasan Bulut & Ege Üniversitesi \\
Hasan Dağ & Kadir Has Üniversitesi \\
Hasan Kurban & Indiana Universitesi \\
Hilal Arslan & Yıldırım Beyazıt Üniversitesi \\
İlker Birbil & University of Amsterdam \\
İsmail Arı & Özyeğin Üniversitesi \\
Jose L. Abellan & Catholic University Saint Anthony Murcia \\
Mehmet Dalkılıç & Indiana University \\
Mehmet Koyutürk & Case Western Reserve University \\
Mehmet Sıddık Aktaş & Yıldız Teknik Üniversitesi \\
Metin Aktulga & Michigan State University \\
Metin Muradoğlu & Koç Üniversitesi \\
\end{tabular}

\begin{tabular}{ll}
Mohammed Alser & ETH Zürich \\
Mücahid Kutlu & TOBB ETÜ \\
Murat Keçeli & Argonne National Laboratory \\
Oğuz Gülseren & Bilkent Üniversitesi \\
Onur Varol & Sabancı Üniversitesi \\
Osman Barış Malcıoğlu & Orta Doğu Teknik Üniversitesi \\
Osman Serdar Gedik & Ankara Yıldırım Beyazıt Üniversitesi \\
Özcan Öztürk & Bilkent Üniversitesi \\
Pınar Duygulu Şahin & Hacettepe Üniversitesi \\
Reha Oğuz Selvitopi & Lawrence Berkeley National Laboratory \\
Şükrü Torun & Ankara Yıldırım Beyazıt Üniversitesi \\
Süleyman Eken & Kocaeli Üniversitesi \\
Tan Nguyen & Lawrence Berkeley National Laboratory \\
Tandaç Furkan Güçlü & Sabancı Üniversitesi \\
Tuğba Önal Süzek & Muğla Sıtkı Koçman Üniversitesi \\
Tuğrul Senger & İzmir Yüksek Teknoloji Enstitüsü 
\end{tabular}


